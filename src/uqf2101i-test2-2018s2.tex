\documentclass[a4paper]{article}

\usepackage[
    a4paper, left=1cm, right=1cm, top=1cm, bottom=1cm, landscape
]{geometry}
\usepackage{multicol}
\usepackage{amsmath}
\usepackage{IEEEtrantools}

\newcommand{\heading}[1]{{\small\textbf{#1}}}
\newcommand{\subheading}[1]{{\scriptsize\textbf{#1}}}

\begin{document}

\scriptsize                         % Small fonts
\pagenumbering{gobble}              % No page numbers
\setlength\parindent{0pt}           % No indents at start of paragraphs
\setlength{\abovedisplayskip}{3pt}  % Less spacing before equations
\setlength{\belowdisplayskip}{3pt}  % less spacing after equations

% TITLE %
\begin{center}
{\large UQF2101I Cheatsheet}\\{for test 2, by ning}
\end{center}

% BODY %
\begin{multicols*}{4}

%% Learning Objectives %%
\heading{Learning Objectives}
\begin{itemize} \itemsep -0.5em
    % WEEKS 4 TO 7
    % COVERS FROM PROBABILITY DISTRIBUTIONS 
    %   TO HYPOTHESIS TESTING, INCLUSIVE
    % WEEK 4 CLASS 2
    \item Random variables
    \item Distribution of probabilities
    \item Taking expectation, $\mathrm{E}$---mean and variance
    \item Normal distribution \& derived distributions
\end{itemize}

%% Random variables %%
\heading{Random variables \& their distributions}
\begin{itemize} \itemsep -0.5em
    \item Variable that can take on one or more values, each of them
        associated with a probability
    \item Can be discrete or continuous
    \item Discrete random variables are described by a probability mass
        function
    \item Continuous random variables are described by a probability
        density function
    \item The area under a probability distribution function (PDF) is 
        always 1, i.e. $$\int^{\infty}_{-\infty} P(x)\ dx = 1$$
    \item The mean, $\mu$ of a PDF is its central tendency; its variance
        is its variability, or dispersion about the mean.

        For discrete random variables,
        \begin{align*}
            \mu &= \sum_{i=1}^n x_i \cdot f(x_i) \\
            \sigma^2 &= \sum_{i=1}^n
                (x_i - \mu)^2 \cdot f(x_i)
        \end{align*}

        For continuous random variables,
        \begin{align*}
            \mu &= \int^\infty_{-\infty} x \cdot f(x)\ dx \\
            \sigma^2 &= \int^\infty_{-\infty} (x - \mu)^2 \cdot f(x)\ dx
        \end{align*}
\end{itemize}

%% Expectation operator%%
\heading{Expectation operator}
\begin{itemize} \itemsep -0.5em
    \item First, note that the mean and variance produce a single number
        from many outcomes using a weighted average
    \item Intuitively, the taking the expected value is a similar
        process of finding a weighted average
    
        Some basic properties of the expectation operator, $\mathrm{E}$,
        for constants $a$, $b$, and random variable $X$,
        \begin{align*}
            \mathrm{E}(b) &= b\\
            \mathrm{E}(aX + b) &= a\mathrm{E}(X) + b\\
            \mathrm{E}(\mathrm{g}(X)) &\neq \mathrm{g}(\mathrm{E}(X))
        \end{align*}
\end{itemize}

%% Normal distribution%%
\heading{Normal distribution}
\begin{itemize} \itemsep -0.5em
    \item Typically notated as $X\sim N(\mu,\ \sigma^2)$; $X$ is a random
        variable that is normally distribution with mean $\mu$ and
        variance $\sigma^2$
    \item The standard normal distribution, $Z$ is defined as
        $$ Z = \frac{X - \mu}{\sigma} \implies Z\sim N(0,\ 1) $$
        and has a shorthand $P(Z \leq z) = \Phi(z)$
\end{itemize}

%% Other distributions %%
\heading{Derived distributions}\\
The normal distribution is used as a basis to generate other important 
distributions\\

\subheading{Log-normal distribution}\\
If $Y = \mathrm{ln}(X)$, where $Y\sim N(\lambda,\ \xi^2)$, then X is
log-normally distributed with mean $\mu$ and variance $\sigma^2$,
\begin{align*}
    X       &\sim \text{Lognormal}(\mu,\ \sigma^2)\\
    \lambda &= \mathrm{ln}\ \mu - \frac{1}{2}\ \xi^2\\
    \xi^2   &= \mathrm{ln}\left ( 1 + \frac{\sigma^2}{\mu^2} \right )
\end{align*}

\subheading{Chi-square ($\chi^2$)}\\
If $X = Z^2$, where $Z$ is the standard normal, i.e. $Z\sim N(0,\ 1)$,
then X is chi-square ($\chi^2$) distributed. The $\chi^2$ distribution
has an additional parameter $k$, the degree of freedom.

% TODO: GC column
% TODO: consider an 'examples' column

\end{multicols*}
\end{document}
