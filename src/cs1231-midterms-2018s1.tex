\documentclass[a4paper]{article}

\usepackage[
    a4paper, left=1cm, right=1cm, top=1cm, bottom=1cm, landscape
]{geometry}
\usepackage{multicol}
\usepackage{amsmath}
\usepackage{amsfonts}       % \mathbb
\usepackage{amssymb}        % \nmid
\usepackage{enumitem}       % [leftmargin=*]
\usepackage{IEEEtrantools}

\newcommand{\heading}[1]{{\small\underline{\textbf{#1}}}}
\newcommand{\subheading}[1]{{\scriptsize\textbf{#1}}}

\begin{document}

\scriptsize                         % Small fonts
\pagenumbering{gobble}              % No page numbers
\setlength\parindent{0pt}           % No indents at start of paragraphs
\setlength{\abovedisplayskip}{3pt}  % Less spacing before equations
\setlength{\belowdisplayskip}{3pt}  % less spacing after equations

% TITLE %
\begin{center}
{\large CS1231 Cheatsheet}\\{for midterms, by ning}
\end{center}

% BODY %
\begin{multicols*}{4}

%% Preface %%
Appendix A of Epp is not covered. Theorems, corollaries, lemmas, etc. not
mentioned in the lecture notes are marked with an asterisk (*).\\

%% Proofs %%
\heading{Proofs} \\

\subheading{Basic Notation}
\begin{itemize}[leftmargin=*] \itemsep -0.5em
    \item $\mathbb{R}$: the set of all real numbers
    \item $\mathbb{Z}$: the set of integers (includes $0$)
    \item $\mathbb{Q}$: the set of rationals
    \item $\exists$:    there exists...
    \item $\exists!$:   there exists a unique...
    \item $\forall$:    for all...
    \item $\in$:        member of...
    \item $\ni$:        such that...
\end{itemize}

\subheading{Proof Types}
\begin{itemize}[leftmargin=*] \itemsep -0.5em
  \item \textbf{By Construction}: finding or giving a set of directions to
    reach the statement to be proven true.
  \item \textbf{By Contraposition}: proving a statement through its logical
    equivalent contrapositive.
  \item \textbf{By Contradiction}: proving that the negation of the statement
    leads to a logical contradiction.
  \item \textbf{By Exhaustion}: considering each case.
  \item \textbf{By Mathematical Induction}: proving for a base case, then an
    induction step.
    \vspace{-1em} % hackhackhack
    \begin{enumerate} \itemsep -0.2em
      \item $P(a)$
      \item $\forall k \in \mathbb{Z}, k \geq a\;(P(k) \rightarrow P(k+1))$
      \item $\forall n \in \mathbb{Z}, n \geq a\;(P(n))$
    \end{enumerate}
  \vspace{-0.5em}
  \item \textbf{By Strong Induction}: mathematical induction assuming $P(k),
    P(k-1), \cdots, P(a)$ are all true.
\end{itemize}

\subheading{Order of Operations}

First $\sim$ (also represented as $\neg$). No priority within $\land$ and
$\lor$, so $p \land q \lor r$ is ambiguous and should be written as
$(p \land q) \lor r$ or $p \land (q \lor r)$. The implication, $\rightarrow$ is
performed last. Can be overwritten by parenthesis.\\

\subheading{Universal \& Existential Generalisation}\\
\textit{`All boys wear glasses'} is written as
  $$\forall x (\text{Boy}(x) \rightarrow \text{Glasses}(x)) $$
If conjunction was used, this statement would be falsified by the existence of a
`non-boy' in the domain of $x$.\\

\textit{`There is a boy who wears glasses'} is written as
  $$\exists x (\text{Boy}(x) \land \text{Glasses}(x)) $$
If implication was used, this statement would true even if the domain of $x$ is
empty.\\

\subheading{Valid Arguments as Tautologies}\\
All valid arguments can be \textit{restated} as tautologies.\\

\subheading{Rules of Inference}\\
Modus ponens
\begin{eqnarray*}
  &p \rightarrow q \\
  &p \\
  &\boldsymbol{\cdot}\; q
\end{eqnarray*}
Modus tollens
\begin{eqnarray*}
  &p \rightarrow q \\
  &\neg q \\
  &\boldsymbol{\cdot}\; \neg p
\end{eqnarray*}
Generalization
\begin{eqnarray*}
  &p\\
  &\boldsymbol{\cdot}\; p \lor q
\end{eqnarray*}
Specialization
\begin{eqnarray*}
  &p \land q\\
  &\boldsymbol{\cdot}\; p
\end{eqnarray*}
Elimination
\begin{eqnarray*}
  &p \lor q\\
  &\neg q\\
  &\boldsymbol{\cdot}\; p
\end{eqnarray*}
Transitivity
\begin{eqnarray*}
  &p \rightarrow q\\
  &q \rightarrow r\\
  &\boldsymbol{\cdot}\; p \rightarrow r
\end{eqnarray*}
Proof by Division into Cases
\begin{eqnarray*}
  &p \lor q\\
  &p \rightarrow r\\
  &q \rightarrow r\\
  &\boldsymbol{\cdot}\; r
\end{eqnarray*}
Contradiction Rule
\begin{eqnarray*}
  &\neg p \rightarrow \textbf{c}\\
  &\boldsymbol{\cdot}\; p
\end{eqnarray*}

\subheading{Universal Rules of Inference}\\
Only modus ponens, modus tollens, and transitivity have universal versions in
the lecture notes.\\

\subheading{Implicit Quantification}\\
The notation $P(x) \implies Q(x)$ means that every element in the truth set of
$P(x)$ is in the truth set of $Q(x)$, or equivalently, $\forall x, P(x)
\rightarrow Q(x)$.\\

The notation $P(x) \iff Q(x)$ means that $P(x)$ and $Q(x)$ have identical truth
sets, or equivalently, $\forall x, P(x) \leftrightarrow Q(x)$.\\

\subheading{Implication Law}\\
$$p \rightarrow q \equiv \neg p \lor q$$

\subheading{Universal Instantiation}\\
If some property is true of everything in a set, then it is true of any
particular thing in the set.\\

\subheading{Universal Generalization}\\
If $P(c)$ must be true, and we have assumed nothing about $c$, then $\forall x,
P(x)$ is true.\\

\subheading{Regular Induction}\\
\begin{eqnarray*}
  &P(0) \\
  &\forall k \in \mathbb{N}, P(k) \rightarrow P(k+1) \\
  & \forall
\end{eqnarray*}

\subheading{Epp T2.1.1 Logical Equivalences}\\
Commutative Laws
  $$ p \land q \equiv q \land p $$
  $$ p \lor  q \equiv q \lor  p $$
Associative Laws
  $$ (p \land q) \land r \equiv p \land (q \land r) $$
  $$ (p \lor  q) \lor  r \equiv p \lor  (q \lor  r) $$
Distributive Laws
  $$ p \land (q \lor  r) \equiv (p \land q) \lor  (p \land r) $$
  $$ p \lor  (q \land r) \equiv (p \lor  q) \land (p \lor  r) $$
Identity Laws
  $$ p \land \textbf{t} \equiv p $$
  $$ p \lor  \textbf{c} \equiv p $$
Negation Laws
  $$ p \lor  \neg p \equiv \textbf{t} $$
  $$ p \land \neg p \equiv \textbf{c} $$
Double Negative Law
  $$ \neg ( \neg p ) \equiv p $$
Idempotent Laws
  $$ p \land p \equiv p $$
  $$ p \lor  p \equiv p $$
Universal Bound Laws
  $$ p \lor  \textbf{t} \equiv \textbf{t} $$
  $$ p \land \textbf{c} \equiv \textbf{c} $$
De Morgan's Laws
  $$ \neg ( p \land q ) \equiv \neg p \lor  \neg q $$
  $$ \neg ( p \lor  q ) \equiv \neg p \land \neg q $$
Absorption Laws
  $$ p \lor  (p \land q) \equiv p $$
  $$ p \land (p \lor  q) \equiv p $$
Negations of $\textbf{t}$ and $\textbf{c}$
  $$ \neg \textbf{t} \equiv \textbf{c} $$
  $$ \neg \textbf{c} \equiv \textbf{t} $$

\subheading{Definition 2.2.1 (Conditional)}\\
If $p$ and $q$ are statement variables, the conditional of $q$ by $p$ is ``if
$p$ then $q$" or ``$p$ implies $q$", denoted $p \rightarrow q$. It is false
when $p$ is true and $q$ is false; otherwise it is true. We call $p$ the
\textit{hypothesis} (or \textit{antecedent}), and $q$ the \textit{conclusion}
(or \textit{consequent}).\\

A conditional statement that is true because its hypothesis is false is called
\textit{vacuously true} or \textit{true by default}.\\

\subheading{Definition 2.2.2 (Contrapositive)}\\
The contrapositive of $p \rightarrow q$ is $\neg q \rightarrow \neg p$.\\

\subheading{Definition 2.2.3 (Converse)}\\
The converse of $p \rightarrow q$ is $q \rightarrow p$.\\

\subheading{Definition 2.2.4 (Inverse)}\\
The inverse of $p \rightarrow q$ is $\neg p \rightarrow \neg q$.\\

\subheading{Definition 2.2.6 (Biconditional)}\\
The biconditional of $p$ and $q$ is denoted $p \leftrightarrow q$ and is true if
both $p$ and $q$ have the same truth values, and is false if $p$ and $q$ have
opposite truth values.\\

\subheading{Definition 2.2.7 (Necessary \& Sufficient)}\\
``$r$ is sufficient for $s$" means $r \rightarrow s$, ``$r$ is necessary for
$s$" means $\neg r \rightarrow \neg s$ or equivalently $s \rightarrow r$.\\

\subheading{Definition 2.3.2 (Sound \& Unsound Arguments)}\\
An argument is called \textit{sound}, iff it is valid and all its premises are
true.\\

\subheading{Definition 3.1.3 (Universal Statement)}\\
A \textit{universal statement} is of the form $$\forall x \in D, Q(x)$$ It is
defined to be true iff $Q(x)$ is true for every $x$ in $D$. It is defined to be
false iff $Q(x)$ is false for at least one $x$ in D.\\

\subheading{Definition 3.1.4 (Existential Statement)}\\
A \textit{existential statement} is of the form $$\exists x \in D \text{ s.t. }
Q(x)$$ It is defined to be true iff $Q(x)$ is true for at least one $x$ in $D$.
It is defined to be false iff $Q(x)$ is false for all $x$ in $D$.\\

\subheading{Theorem 3.2.1 (Negation of Universal State.)}\\
The negation of a statement of the form $$\forall x \in D, P(x)$$ is logically
equivalent to a statement of the form $$\exists x \in D \text{ s.t. } \neg
P(x)$$

\subheading{Theorem 3.2.2 (Negation of Existential State.)}\\
The negation of a statement of the form $$\exists x \in D \text{ s.t. } P(x)$$
is logically equivalent to a statement of the form $$\forall x \in D, \neg
P(x)$$

\heading{Number Theory} \\

\subheading{Properties (of Numbers)} \\
Closure, i.e.
$$\forall x, y \in \mathbb{Z},\;
         x + y \in \mathbb{Z},\text{ and }
            xy \in \mathbb{Z}$$
Commutativity, i.e.
  $$a+b=b+a\text{ and }ab=ba$$
Distributivity, i.e.
  $$a(b+c) = ab + ac \text{ and } (b+c)a = ba + ca$$
Trichotomy, i.e.
  $$(a < b) \oplus (b < a) \oplus (a = b)$$
(Can be used without proof)\\

\subheading{Definition 1.1.1 (Colorful)}

An integer $n$ is said to be colorful if there exists some integer $k$ such
that $n = 3k$.\\

\subheading{Definition 1.3.1 (Divisibility)}\\
If $n$ and $d$ are integers and $d \neq 0$,
  $$ d|n \iff \exists k \in \mathbb{Z} \text{ s.t. } n=dk $$

\subheading{Proposition 1.3.2 (Linear Combination)}
$$\forall a, b, c \in \mathbb{Z},\;
    a | b \land a | c \rightarrow
    \forall x, y \in \mathbb{Z},\;
    a | (bx + cy)$$
If $a$ divides $b$ and $c$, then it also divides their linear combination
$(bx + cy)$.\\

\subheading{Theorem 4.1.1 (Linear Combination)}\\
$$\forall a,b,c \in \mathbb{Z},\;
  a | b \land a | c \rightarrow \forall x,y \in \mathbb{Z}, a | (bx + cy)$$

\subheading{Epp T4.3.3 (Transitivity of Divisibility)}
$$\forall a, b, c \in \mathbb{Z},\;
  a | b \land b | c \rightarrow a | c$$

\subheading{Theorem 4.4.1 (Quotient-Remainder Theorem)}
Given any integer $a$ and any positive integer $b$, there exist unique integers
$q$ and $r$ such that $$ a = bq + r \text{ and } 0 \leq r < b $$

\subheading{Representation of Integers}\\
Given any positive integer $n$ and base $b$, repeatedly apply the
Quotient-Remainder Theorem to get,
\begin{eqnarray*}
  n   &= bq_0 + r_0 \\
  q_0 &= bq_1 + r_1 \\
  q_1 &= bq_2 + r_2 \\
  & \cdots \\
  q_{m-1} &= bq_m + r_m
\end{eqnarray*}

The process stops when $q_m = 0$. Eliminating the quotients $q_i$ we get,
  $$ n = r_mb^m + r_{m-1}b^{m-1} + \cdots r_1b + r_0 $$

Which may be represented compactly in base $b$ as a sequence of the digits
$r_i$,
  $$ n = (r_m r_{m-1} \cdots r_1 r_0)_b $$

\subheading{Definition 4.2.1 (Prime number)}\\
\begin{IEEEeqnarray*}{rCl}
  n\text{ is prime } &\iff& \forall r, s \in \mathbb{Z}^+ \\
                    &&n = rs \rightarrow \\
                    &&(r=1 \land s=n) \lor (r=n \land s=1) \\
  n\text{ is composite } &\iff& \exists r, s \in \mathbb{Z}^+ \text{ s.t. }\\
                    &&n = rs\;\land \\
                    &&(1 < r < n) \land (1 < s < n)
\end{IEEEeqnarray*}

\subheading{List of Primes to 100}\\
2, 3, 5, 7, 11, 13, 17, 19, 23, 29, 31, 37, 41, 43, 47, 53, 59, 61, 67, 71, 73,
79, 83, 89, 97.\\

\subheading{Proposition 4.2.2}\\
For any two primes $p$ and $p'$,
  $$p\;|\;p' \rightarrow p = p'$$

\subheading{Theorem 4.2.3}\\
If $p$ is a prime and $x_1, x_2, \cdots, x_n$ are any integers s.t.
$p\;|\;x_1x_2\cdots x_n$, then $p\;|\;x_i$ for some $x_i, i \in \{1, 2, \cdots,
n\}$.\\

\subheading{Epp T4.3.5 (Unique Prime Factorisation)}\\
Given any integer $n > 1$
\begin{IEEEeqnarray*}{rCl}
  \exists k                  &\in& \mathbb{Z}^+, \\
  \exists p_1,p_2,\cdots,p_k &\in& \text{ primes}, \\
  \exists e_1,e_2,\cdots,e_k &\in& \mathbb{Z}^+,
\end{IEEEeqnarray*}
such that $$n=p_1^{e_1} p_2^{e_2} \cdots p_k^{e_k}$$
and any other expression for $n$ as a product of prime numbers is identical,
except perhaps for the order in which the factors are written.\\

\subheading{Epp Proposition 4.7.3}\\
For any $a \in \mathbb{Z}$ and any prime $p$,
  $$ p\;|\;a \rightarrow p \nmid (a+1) $$

\subheading{Epp T4.7.4 (Infinitude of Primes)}\\
The set of primes is infinite.\\

\subheading{Definition 4.5.4 (Relatively Prime)}\\
Integers $a$ and $b$ are \textit{relatively prime} (or \textit{coprime}) iff
$\mathrm{gcd}(a,b)=1$.\\

\subheading{Definition 4.3.1 (Lower Bound)}\\
An integer $b$ is said to be a \textit{lower bound} for a set $X \subseteq
\mathbb{Z}$ if $b \leq x$ for all $x \in X$.\\

Does not require $b$ to be in $X$.\\

\subheading{Theorem 4.3.2 (Well Ordering Principle)}\\
If a non-empty set $S \subseteq \mathbb{Z}$ has a lower bound, then $S$ has a
least element.\\

Note three conditions: $|S| > 0$, $S \subseteq \mathbb{Z}$, and $S$ has lower
bound.\\

Likewise, if ... upper bound ... has a greatest element.\\

\subheading{Proposition 4.3.3 (Uniqueness of least element)}\\
If a set $S$ has a least element, then the least element is unique.\\

\subheading{Proposition 4.3.4 (Uniqueness of greatest e.)}\\
If a set $S$ has a greatest element, then the greatest element is unique.\\

\subheading{Theorem 4.4.1 (Quotient-Remainder Theorem)}\\
Given any integer $a$ and any positive integer $b$, there exist unique integers
$q$ and $r$ such that $$ a = bq + r \text{ and } 0 \leq r < b$$

\subheading{Definition 4.5.1 (Greatest Common Divisor)}\\
Let $a$ and $b$ be integers, not both zero. The \textit{greatest common divisor}
of $a$ and $b$, denoted $\mathrm{gcd}(a, b)$, is the integer $d$ satisfying

\begin{enumerate} \itemsep -0.5em
  \item $d\;|\;a$ and $d\;|\;b$
  \item $\forall c \in \mathbb{Z}\;((c\;|\;a)\ \land (c\;|\;b) \rightarrow c \leq d)$
\end{enumerate}

\subheading{Proposition 4.5.2 (Existence of gcd)}\\
For any integers $a$, $b$, not both zero, their gcd exists and is unique.\\

\subheading{Theorem 4.5.3 (B\'ezout's Identity)}\\
Let $a$, $b$ be integers, not both zero, and let $d = \mathrm{gcd}(a, b)$. Then
there exists integers $x$, $y$ such that $$ax + by = d$$

Or, the gcd of two integers is some linear combination of the said numbers,
where $x$, $y$ above have multiple solution pairs once a solution pair $(x, y)$
is found. Also solutions, for any integer $k$,

$$ (x+\frac{kb}{d}, y-\frac{ka}{d}) $$

\subheading{*Epp T8.4.8 (Euclid's Lemma)}\\
For all $a, b, c \in \mathbb{Z}$, if $\mathrm{gcd}(a, c) = 1$ and $a\;|\;bc$,
then $a\;|\;b$.\\

\subheading{*Epp Lemma 4.8.2}\\
If $a, b \in \mathbb{Z}^+$, and $q, r \in \mathbb{Z}$ s.t. $a = bq + r$, then

$$\mathrm{gcd}(a, b)= \mathrm{gcd}(b, r)$$

\subheading{Proposition 4.5.5}\\
For any integers $a$, $b$, not both zero, if $c$ is a common divisor of $a$ and
$b$, then $c\;|\;\mathrm{gcd}(a,b)$.\\

% TODO: LCM for finals

\subheading{Definitoin 4.7.1 (Congruence modulo)}\\
Let $m, z \in \mathbb{Z}$ and $d \in \mathbb{Z}^+$. We say that $m$ is
\textit{congruent} to $n$ \textit{modulo} $d$ and write

$$ m \equiv n\ (\mathrm{mod}\; d) $$

iff

$$ d\;|\;(m-n) $$

More concisely,

$$ m \equiv n\ (\mathrm{mod}\; d) \iff d\;|\;(m-n) $$

\subheading{Epp T8.4.1 (Modular Equivalences)}\\
Let $a, b, n \in \mathbb{Z}$ and $n > 1$. The following statements are all
equivalent,
\begin{enumerate} \itemsep -0.5em
    \item $n\;|\;(a-b)$
    \item $a \equiv b\ (\mathrm{mod}\; n)$
    \item $a = b + kn$ for some $k \in \mathbb{Z}$
    \item $a$ and $b$ have the same non-negative remainder when divided by $n$
    \item $a\;\mathrm{mod}\;n = b\;\mathrm{mod}\;n$
\end{enumerate}

\subheading{Epp T8.4.3 (Modulo Arithmetic)}\\
Let $a, b, c, d, n \in \mathbb{Z}$, $n > 1$, and suppose\\

{\centering
$a \equiv c\ (\mathrm{mod}\; n)$ and $b \equiv d\ (\textrm{mod}\; n)$\\
}

Then

\begin{enumerate} \itemsep -0.5em
    \item $(a + b) \equiv (c + d)\ (\mathrm{mod}\;n)$
    \item $(a - b) \equiv (c - d)\ (\mathrm{mod}\;n)$
    \item $ab \equiv cd\ (\mathrm{mod}\;n)$
    \item $a^m \equiv c^m\ (\mathrm{mod}\;n)$, for all $m \in \mathbb{Z}^+$
\end{enumerate}

\subheading{Epp Corollary 8.4.4}\\
Let $a, b, c, d, n \in \mathbb{Z}$, $n > 1$, then

$$ ab \equiv [(a\;\mathrm{mod}\;n)(b\;\mathrm{mod}\;n)]\ (\mathrm{mod}\;n) $$

or equivalently,

$$ ab\;\mathrm{mod}\;n = [(a\;\mathrm{mod}\;n)(b\;\mathrm{mod}\;n)]\ \mathrm{mod}\;n $$

In particular, if $m$ is a positive integer, then

$$ a^m \equiv [(a\;\textrm{mod}\;n)^m]\ (\mathrm{mod}\;n) $$

\subheading{Definition 4.7.2 (Multiplicative inv. modulo $n$)}\\
For any integers $a, n$ with $n > 1$, if an integer $s$ is such that $as \equiv
1\ (\mathrm{mod}\;n)$, then $s$ is the \textit{multiplicative inverse of $a$
modulo $n$}. We may write $s$ as $a^{-1}$.\\

Because the commutative law still applies in modulo arithmetic, we also have

$$a^{-1}a \equiv 1\ (\mathrm{mod}\;n)$$

Multiplicative inverses are not unique. If $s$ is an inverse, then so is $(s +
kn)$ for any integer $k$.\\

\subheading{Theorem 4.6.3 (Existence of multiplicative inverse)}\\
For any integer $a$, its multiplicative inverse modulo $n$ where $n>1$,
$a^{-1}$, exists iff $a$ and $n$ are coprime.\\

\subheading{Finding the Multiplicative Inverse}\\
For example, to find the multiplicative inverse of $5\ \textrm{mod}\; 18$,
\begin{IEEEeqnarray*}{rCl}
  18 = 3 &\times& 5 + 3 \\
   5 = 1 &\times& 3 + 2 \\
   3 = 1 &\times& 2 + 1 \\
   1 = 1 &\times& 1 + 0
\end{IEEEeqnarray*}
So
\begin{IEEEeqnarray*}{rCl}
  1 &=& 1 \times 1 + 0 = 1 \\
    &=& 1(3 - 1 \times 2) = 3 - 2  \\
    &=& 3 - (5 - 3) = 2 \times 3 - 5 \\
    &=& 2(18 - 3 \times 5) - 5 = 2 \times 18 - 7 \times 5 \\
  1 - 2 \times 18 &=& -7 \times 5 \\
  1 - 2 \times 18 &\equiv& -7 \times 5\ (\mathrm{mod}\; 18) \\
                1 &\equiv& -7 \times 5\ (\mathrm{mod}\; 18)
\end{IEEEeqnarray*}
Therefore, we have $5^{-1}\ \textrm{mod}\; 18 = -7$, or equivalently under
modulo $11$.\\

\subheading{Corollary 4.7.4 (Special case: $n$ is prime)}\\
If $n=p$ is a prime number, then all integers $a$ in the range $0<a<p$ have
multiplicative inverses modulo $p$.\\

\subheading{Epp T8.4.9 (Cancellation Law for mod. arith.)}\\
For all $a, b, c, n \in \mathbb{Z}$, $n>1$, and $a$ and $n$ are coprime,

$$ ab \equiv ac\ (\mathrm{mod}\;n) \rightarrow b \equiv c\ (\mathrm{mod}\;n) $$

\end{multicols*}
\end{document}
