\documentclass[a4paper]{article}
\usepackage[
    a4paper, left=1cm, right=1cm, top=1cm, bottom=1cm, landscape
]{geometry}

\usepackage{amsmath}
\usepackage{amssymb}
\usepackage{enumitem}
\usepackage{multicol}
\usepackage{IEEEtrantools}
\usepackage{graphicx}

% heading macros
\newcommand{\heading}[1]{{\small\underline{\textbf{#1}}}\smallskip}
\newcommand{\subheading}[1]{{\scriptsize\textbf{#1}}}
\renewenvironment{section}[1]
  {
    \subheading{#1}

  }{
    \smallskip
  }

% math macros
\allowdisplaybreaks
\newcommand{\independent}{\raisebox{0.05em}{\rotatebox[origin=c]{90}{$\models$}}}
\newcommand{\expectation}[1]{\operatorname{E}[#1]}
\newcommand{\ddx}{\frac{d}{dx}}
\DeclareMathOperator{\Rank}{Rank}
\DeclareMathOperator{\Var}{Var}
\DeclareMathOperator{\MSE}{MSE}
\DeclareMathOperator{\eff}{eff}
\DeclareMathOperator{\Cov}{Cov}

% itemize settings
\renewcommand\labelitemi{-}
\setlist[itemize]{leftmargin=*, noitemsep, nosep}

\begin{document}

\scriptsize                         % Small fonts
\pagenumbering{gobble}              % No page numbers
\setlength\parindent{0pt}           % No indents at start of paragraphs
\setlength{\abovedisplayskip}{3pt}  % Less spacing before equations
\setlength{\belowdisplayskip}{3pt}  % less spacing after equations

% TITLE %
\begin{center}
  {\large ST2132 Cheatsheet}\\{for finals, by ning}
\end{center}

% BODY %
\begin{multicols*}{4}

\heading{ANOVA}

In this section, $I$ denotes the treatments/groups, and $J$ the measurements in
each group. For two-factor, $I, J$ denote the treatments/groups, and $J$ the
measurements within each $I, J$ combination.
\begin{align*}
  \sum^I_{i=1}\sum^J_{j=1} (Y_{ij} - \bar{\bar{Y}})^2
    &= \sum^I_{i=1}\sum^J_{j=1} (Y_{ij} - \bar{Y})^2 + \\
    &\quad J\sum^I_{i=1} (\bar{Y_i} - \bar{\bar{Y}})^2 \\
  SS_{TOT} &= SS_W + SS_B \\
  SS_B/(I-1) &\sim \chi^2_{I-1} \\
  SS_W/[I(J-1)] &\sim \chi^2_{I(J-1)} \\
  \expectation{SS_W} &= I(J-1)\sigma^2 \\
  \expectation{SS_B} &= J\sum^I_{i=1} \alpha_i^2 + (I-1)\sigma^2
\end{align*}

\begin{section}{One-factor ANOVA, same-sized groups}
  The test statistic is
  $$F = \frac{SS_B/(I-1)}{SS_W/[I(J-1)]}
    \sim F_{I-1, I(J-1)}$$
  Reject $H_0$ if  $F > F_{I-1, I(J-1)}(\alpha)$.
\end{section}

\begin{section}{One-factor ANOVA, differently-sized groups}
  The test statistic follows a slightly different degree of freedoms $I-1 :=
  df_1$ and $\sum^n_{i=1} J_i - I =: df_2$.
  $$F \sim F_{df_1, df_2}$$
\end{section}

\begin{section}{Two-factor ANOVA}
  There will be an additional sum of squares term for the interaction between
  groups. Its associated degree of freedom in as a chi-square distributed random
  variable, and within the final $F$ distributed test statistic is $(I-1)(J-1)$.
  The sum of squared errors (within groups) has degree of freedom as chi-squared
  $IJ(K-1)$.
\end{section}

\begin{section}{Post-ANOVA Tests}
  Turkey's correction and Bonferroni's correction reduces the probability of
  type I error in multiple tests after the ANOVA. The Kruskal-Wallis test, a
  generalisation of the Mann-Whitney test, is a nonparametric test which is
  particularly useful for small data sets.
\end{section}

\heading{Two-Sample Tests}

In this section, $X_1, \cdots, X_n$ and $Y_1, \cdots, Y_m$ are each i.i.d.
samples of $X$ and $Y$ respectively, unless otherwise stated. Define $D_i := X_i
- Y_i$. The variances for $X$ and $Y$ are unknown. \smallskip

\begin{section}{Estimating the Equality of Variances}
  If $S_X \leq 2S_Y$ or $S_Y \leq 2S_X$, it is reasonable to assume that
  $\sigma_X = \sigma_Y$.
\end{section}

\begin{section}{Normal, Unpaired, with Equal Variance}
  Calculate the pooled variance, $s_p^2$ as such.
  $$S_p^2 = \frac{(n-1)S_X^2 + (m-1) S_Y^2}{m + n - 2}$$
  The test statistic $t$ is
  $$t = \frac{(\bar{X}-\bar{Y}) - (\mu_X - \mu_Y)}{SE_{\bar{X}-\bar{Y}}}$$
  which follows a $t$ distribution with $m + n - 2$ degrees of freedom.
\end{section}

\begin{section}{Normal, Unpaired, with Unequal Variance}
  The variance of the sampling distribution $\Var(\bar{X}-\bar{Y})$ is simply
  $$\Var(\bar{X}-\bar{Y}) = S^2_X/n + S^2_Y/m$$
  The test statistic $t$ is
  $$t = \frac{\bar{X}-\bar{Y} - (\mu_X-\mu_Y)}{\sqrt{S^2_X/n + S^2_Y/m}}$$
  which follows a $t$ distribution with degrees of freedom $df$ as
  $$df = \frac{(S^2_X/n + S^2_Y/m)^2}{
    \frac{(S^2_X/n)^2}{n-1} + \frac{(S^2_Y/m)^2}{m-1}
  }$$
\end{section}

\begin{section}{Normal, Paired}
  The variance of the sampling distribution $\Var(\bar{D})$ can be estimated
  simply by the unbiased sample variance of the series of $D_i$ random
  variables. The test statistic $t$ is
  $$t = \frac{\bar{D} - \mu_D}{SE_{\bar{D}}}$$
\end{section}

\begin{section}{Unpaired, Nonparametric}
  Rank the values of the samples $X_i, Y_j$ from 1 to $n + m$. The null
  hypotheses is that $X_i, Y_j$ are distributed identically, and therefore
  should rank `evenly'. Then, the rank sum scores are defined as
  $$R_X = \sum^n_{i=1} \Rank(X_i); \quad
    R_Y = \sum^m_{j=1} \Rank(Y_j)$$
  Select the sample with the smaller size (w.r.t. $n, m$). Denote its rank sum
  score $R$, and define $R' := n(n+m+1) - R$. The Mann-Whitney test statistic is
  $$R^* = \min(R, R')$$
  $H_0$ is rejected for small $R^*$.
\end{section}

\begin{section}{Paired, Parametric}
  Rank the \textit{absolute} values of $D_i$ from 1 to $n=m$. The null
  hypotheses is that $D$ is distributed symmetrically about 0. Define
  \begin{align*}
    W_+ &= \sum \{\Rank(D_i) | D_i > 0 \} \\
    W_- &= \sum \{\Rank(D_i) | D_i < 0 \}
  \end{align*}
  Denote $W := \min(W_-, W_+)$. $H_0$ is rejected for small $W$.
\end{section}

\heading{Hypothesis Testing}

In a hypothesis testing question, you must include (i) assumptions made, (ii)
the null and alternate hypotheses, (iii) the test statistic and its
distribution, (iv) the $p$-value, and (v) the conclusion. \smallskip

\begin{section}{Terminology}
  \begin{itemize}
    \item The significance level (or size) $\alpha$ of a test is the probability
      of committing a \textit{type I} error, or rejecting the null hypothesis,
      $H_0$ when it is true.
    \item The power $1-\beta$ of a test is the probability that $H_0$ is
      rejected when it is false.
    \item $\beta$ denotes the probability of a \textit{type II} error, or
      failing to reject $H_0$ when it is false.
    \item The $\alpha$ and power of a tests are mutual trade-offs.
    \item The set of values of a test statistic leading to rejection of $H_0$ is
      the rejection or critical region. Those leading to acceptance is the
      acceptance region.
    \item The $p$-value is the smallest significance level at which $H_0$ would
      be rejected.
    \item The null distribution is the probability distribution of the test
      statistic when $H_0$ is true.
  \end{itemize}
\end{section}

\begin{section}{Simple and Composite Hypotheses}
  A hypothesis that does not completely specify the probability distribution is
  called a composite hypothesis. Otherwise, it is a simple hypothesis. A
  hypothesis that is `one-tailed' is called a `one-sided' alternative.
\end{section}

\begin{section}{Uniformly Most Powerful}
  If an alternative hypothesis $H_1$ is composite, a test that is most powerful
  for every simple alternative in $H_1$ is said to be uniformly most powerful.
  The test which is uniformly most powerful for a one-sided alternative is not
  for the two-sided.
\end{section}

\begin{section}{Confidence Interval}
  Denote the acceptance region of the test as $A(\theta_0)$. Then, the set
  $$C(X) = \{\theta | X \in A(\theta) \}$$
  is a $100\%(1-\alpha)$ confidence region for $\theta$. The CI contains all the
  values of $\theta$ for which the null hypothesis $H_0: \theta = \theta_0$ is
  not rejected.
\end{section}

\begin{section}{Neyman-Pearson Lemma}
  Suppose that $H_0$ and $H_1$ are simple hypotheses. Set the significance level
  of the test at $\alpha$. Any other test for which the significance level is
  less than or equal to $\alpha$ has power less than or equal to that of the
  likelihood ratio test.
\end{section}

\begin{section}{Likelihood Ratio Test (LRT)}
  In the case of the simple alternative hypothesis, simply define $\Lambda$
  directly.
  $$\Lambda = \frac{
    L(\theta | H_0)
  }{
    L(\theta | H_1)
  }$$
\end{section}

\begin{section}{Generalised Likelihood Ratio Test (GLRT)}
  The generalised likelihood ratio test a non-optimal test used for situations
  of composite hypothesis where no optimal test exists. Denote the null and
  alternative hypotheses as $H_0: \theta \in \omega_0$ and $H_1: \theta \in
  \omega_1$ respectively, where $\omega_0, \omega_1$ are disjoint and subsets of
  $\Omega$, the sample space. The generalised likelihood ratio test statistic is
  $$\Lambda^* = \frac{
    \max_{\theta \in \omega_0} L(\theta)
  }{
    \max_{\theta \in \omega_1} L(\theta)
  }$$
  For simplicity, we define $\Lambda$ such that $\Lambda = \min(\Lambda^*, 1)$.
  $$\Lambda = \frac{
    \max_{\theta \in \omega_0} L(\theta)
  }{
    \max_{\theta \in \Omega} L(\theta)
  }$$
  Then, the generalised likelihood test rejects for $\Lambda \leq \lambda_0$,
  where $P(\Lambda \leq \lambda_0 | H_0) = \alpha$.
\end{section}

\begin{section}{Distribution of $-2\log \Lambda$}
  For the GLRT, As the sample size $n \rightarrow \infty$, Under smoothness
  conditions on the pmfs or pdfs involved, the null distribution of $-2\log
  \Lambda$ tends to a chi-square distribution with degrees of freedom $df$ as
  $$df = \dim \Omega - \dim \omega_0$$
  $\dim \Omega, \dim \omega_0$ are the number of free parameters under $\Omega$
  and $\omega_0$ respectively. E.g. a simple hypothesis has no free parameters,
  but a composite alternative hypothesis such as $H_1: \mu_0 \neq 0$
  \textit{implies} that $\mu$ is free in $\Omega$, and so $\dim \Omega = 1$. For
  example,
  $$\Lambda = \prod^m_{i=1} \left(
    \frac{p_i(\hat{\theta})}{\hat{p_i}}
  \right )^{x_i}$$
  Denotes $\Lambda$ for the multinomial distribution with values $x_1, \cdots,
  x_m$. The numerator under $H_0$ is maximised by the MLE $\hat{\theta}$, while
  the unrestricted denominator is maximised by individual MLEs $\hat{p_i}$.
\end{section}

\begin{section}{Pearson Chi-square Test}
  The Pearson chi-square test is asymptotically equal to the GLRT. The test
  statistic for a multinomial distributed r.v. is
  $$X^2 = \sum^m_{i=1}\frac{(O_i - E_i)^2}{E_i}
    = \sum^m_{i=1} \frac{(x_i - np_i(\hat{\theta}))^2}{np_i(\hat{\theta})}$$
  Where $X^2 \sim \chi^2_{m-k-1}$, $k$ is the number of values of the
  multinomial distribution.
\end{section}

\heading{Efficiency \& Sufficiency}

\begin{section}{Mean Square Error (MSE)}
  The MSE is a common measure of accuracy of an estimator.
  \begin{align*}
    \MSE(\hat{\theta}) &= \expectation{(\hat{\theta}-\theta_0)^2} \\
                       &= \Var(\hat{\theta}) +
                          (\expectation{\hat{\theta}} - \theta_0)^2 \\
                       &= \text{SE}^2 + \text{bias}^2
  \end{align*}
\end{section}

\begin{section}{Efficiency}
  The efficiency of two estimators, $\hat{\theta}_0, \hat{\theta}_1$ is given as
  $$\eff(\hat{\theta}_0, \hat{\theta}_1) :=
    \Var(\hat{\theta}_1)/\Var(\hat{\theta}_0)$$
  When any of the $\Var(\hat{\theta})$ is estimated via the asymptotic variance,
  the efficiency is called the asymptotic relative efficiency.
\end{section}

\begin{section}{Cram\'er-Rao Inequality}
  Under smoothness assumptions of a $f(x|\theta)$ for a statistic $T:=t(X_1,
  \cdots, X_n)$
  $$\Var(T) \geq \frac{1}{nI(\theta)}$$
  This gives the lower bound for the variance of any estimator of $\theta$. An
  unbiased estimator whose variance achieves this lower bound is said to be
  efficient. The MLE is asymptotically efficient.
\end{section}

\begin{section}{Sufficiency}
  A statistic $T(X_1, \cdots, X_n)$ is said to be sufficient for $\theta$ if the
  conditional distribution of $X_1, \cdots, X_n$ given $T=t$ does not depend on
  $\theta$ for any value of $t$. If $T$ is sufficient for $\theta$, the MLE for
  $\theta$ is a function only of $T$.
\end{section}

\begin{section}{Factorization Theorem}
  The statistic $T(X_1, \cdots, X_n)$ is sufficient for a parameter $\theta$ iff
  the joint pdf factorises in the form
  $$f(\vec{x}|\theta) = g(T(\vec{x}), \theta)h(\vec{X})$$
\end{section}

\begin{section}{Exponential Family of Probability Distributions}
  1-parameter members of the exponential family have pdfs or pmfs in the form
  $$f(x|\theta) = \begin{cases}
      \exp\{c(\theta)T(x)+d(\theta)+S(x)\}, & \text{if } x \in A, \\
      0, & \text{otherwise.}
  \end{cases}$$
  where the set $A$ does not depend on $\theta$.
\end{section}

\begin{section}{Rao-Blackwell Theorem}
  Let $\hat{\theta}_0$ be an estimator for $\theta$ with finite second moment,
  $T$ a sufficient statistic for $\theta$, and $\hat{\theta}_1 =
  \expectation{\hat{\theta}_0|T}$.
  $$\expectation{(\hat{\theta}_1-\theta)^2} \leq
    \expectation{(\hat{\theta_0}-\theta)^2}$$
  $\hat{\theta}_1$ is an estimator of $\theta$ which is better than any
  estimator $\hat{\theta}_0$ since
  $\hat{\theta}_1=\expectation{\hat{\theta}_0|T}$ which is a function of the
  sufficient statistic $T$.
\end{section}

\heading{Other Stuff}

% Derivative Rules
\begin{IEEEeqnarray*}{rccl}
  \ddx \; & f(g(x))           & \; = \; & f'(g(x))g'(x) \\
  \ddx \; & f(x)g(x)          & \; = \; & f(x)g'(x) + f'(x)g(x) \\
  \ddx \; & \frac{f(x)}{g(x)} & \; = \; &
    \frac{f'(x)g(x) - f(x)g'(x)}{g(x)^2} \\
  \ddx \; & f(x)^{g(x)}       & \; = \; &
    f(x)^{g(x)}\left (g'(x)\ln f(x) + g(x)\frac{f'(x)}{f(x)} \right ) \\
  \int^b_a & u\, dv           & \; = \; &
    uv\biggr|^b_a - \int^b_a v\, du
\end{IEEEeqnarray*}

% Gamma Function
\begin{align*}
  \Gamma(z+1) &= z\Gamma(z) \\
  \Gamma(1) &= 1 \\
  \Gamma(n) &= 1 \cdot 2 \cdot \cdots \cdot (n-1) = (n-1)!
\end{align*}

% Expectation and Variance identities
\begin{align*}
  \expectation{X} &= \sum_i x_i p(x_i) \\
  \expectation{X} &= \int^\infty_{-\infty} xf(x)\ dx \\
  \expectation{Y} &= \expectation{\expectation{Y|X}} \\
  \Var(X) &= \expectation{ (X- \expectation{X})^2 } \\
  \Var(X) &= \expectation{X^2} - \expectation{X}^2 \\
  \Var(a + bX) &= b^2Var(X) \\
  \Var(Y) &= \Var(\expectation{Y|X}) + \expectation{\Var(Y|X)} \\
  \Cov(X, Y) &= \expectation{(X-\mu_X)(Y-\mu_Y)} \\
  \Cov(X, Y) &= \expectation{XY} - \expectation{X}\expectation{Y} \quad
    \text{if }X\independent Y \\
  \rho &= \frac{\Cov(X, Y)}{\sqrt{\Var(X)\Var(Y)}} \\
  S^2 &= \frac{1}{n-1} \sum^n_{i=1} (X_i - \bar{X})^2 \\
  S^2_n &= \frac{1}{n}\sum^n_{i=1} (X_i - \bar{X})^2
\end{align*}
\begin{align*}
    &\;\Cov(aX+bY, cW+dV) \\
  =&\;ac\Cov(X,W) + ad\Cov(X,V) + \\
  &\;bc\Cov(Y,W) + bd\Cov(Y,V)
\end{align*}
\begin{equation*}
P(B_j|A) = \frac{P(A|B_j) P(B_j)}{\sum^n_{j=1} P(A|B_j)P(B_j)}
\end{equation*}

% Assorted Distributions

\begin{section}{Information on Various Distributions}
  Binomial:
  \begin{align*}
    p(k) &= \binom{n}{k} p^k (1-p)^{n-k}, \quad k=0,1,\cdots,n \\
    \expectation{X} &= np \\
    \Var(X) &= np(1-p)
  \end{align*}
  Geometric:
  \begin{align*}
    p(k) &= p(1-p)^{k-1}, \quad k=1,\cdots \\
    \expectation{X} &= 1/p \\
    \Var(X) &= (1-p)/p^2
  \end{align*}
  Negative binomial:
  \begin{align*}
    p(k) &= \binom{k-1}{r-1} p^r (1-p)^{k-r}, \quad k=r,r+1,\cdots \\
    \expectation{X} &= r/p \\
    \Var(X) &= [r(1-p)]/p^2
  \end{align*}
  Poisson:
  \begin{align*}
    p(k) &= (\lambda^k e^{-\lambda})/k!, \quad k=0,1,\cdots \\
    \expectation{X} &= \Var(X) = \lambda
  \end{align*}
  Normal:
  \begin{align*}
    f(x) &= \frac{1}{\sigma\sqrt{2\pi}}\exp\{-\frac{1}{2\sigma^2}(x-\mu)^2\},
      \quad -\infty < x < \infty \\
    \expectation{X} &= \mu \\
    \Var(X) &= \sigma^2
  \end{align*}
  Gamma:
  \begin{align*}
    f(x) &= \frac{\lambda^\alpha}{\Gamma(\alpha)} x^{\alpha-1} e^{-\lambda x},
      \quad x \geq 0 \\
    \expectation{X} &= \alpha / \lambda \\
    \Var(X) &= \alpha / \lambda^2
  \end{align*}
  Chi-square:
  \begin{align*}
    Z &\sim N(0, 1) \\
    Z^2 &\sim \chi^2_1 \sim \Gamma(1/2, 1/2) \\
    Y_i \sim \chi^2_1;\; Y_1 + \cdots + Y_n &\sim \chi^2_n, \quad \independent Y_i
  \end{align*}
  $t$:
  \begin{align*}
    \frac{Z}{\sqrt{U/n}} &\sim t_n, \quad Z \sim N(0, 1);\; U \sim \chi^2_n \\
    f(t) &= \frac{\Gamma[(n+1)/2]}{\sqrt{n\pi}\Gamma(n/2)}
      \left ( 1 + \frac{t^2}{n} \right )^{-(n+1)/2}
  \end{align*}
  $F$:
  $$\frac{U/n}{V/m} \sim F_{n, m}, \quad U \sim \chi^2_n; V \sim \chi^2_m$$
  {\tiny
  $$f(x) = \frac{\Gamma[(n+m)/2]}{\Gamma(m/2)\Gamma(n/2)}
    \left ( \frac{n}{m} \right )^{\frac{n}{2}}
    x^{\frac{n}{2}-1}
    \left (1 + \frac{n}{m}x \right )^{-\frac{n+m}{2}}$$
  }
  $f(x)$ is over $x \geq 0$. If $T \sim t_n$ then $T^2 \sim F_{1, n}$.

\end{section}

\begin{section}{Central Limit Theorem}
  For $S_n = \sum^n_{i=1} X_i$,
  $$\lim_{x\rightarrow \infty}
    P \left ( \frac{S_n}{\sigma \sqrt{n}} \leq x \right )
    = \Phi(x)$$
  $$\bar{X} \sim N(\mu, \sigma^2/n)$$
\end{section}

\begin{section}{Linear Functions of a Random Variable}
  Let $Y = g(X)$. To find $f_Y(y)$,
  \begin{align*}
    F_Y(y) &= P(Y \leq y) \\
      &= P(g(X) \leq y) \\
      &= P(X \leq g^{-1}(y)) \\
      &= F_X(g^{-1}(y)) \\
    f_Y(y) &= \frac{d}{dy} F_X(g^{-1}(y)) \\
      &= \frac{dg^{-1}}{dy} f_X(g^{-1}(y))
  \end{align*}
\end{section}

\begin{section}{Non-linear Functions of Random Variables}
  Let $Y = g(\vec{X})$, where $\vec{X} := (X_1, X_2, \cdots)$ with mean vector
  $\vec{\mu}$. Then, in order to find the mean and variance of $Y$, first take
  the Taylor expansion of $g(\vec{X})$,
  \begin{align*}
    Y &= g(\vec{X}) \\
      &\approx
        g(\mu) +
        (X_1 - \mu_1)\frac{\partial g(\mu)}{\partial x_1} +
        (X_2 - \mu_2)\frac{\partial g(\mu)}{\partial x_2} + \cdots
  \end{align*}
  Then, $\expectation{Y} \approx g(\mu)$, and
  $$Var(Y) \approx Var(g(\mu) + (X_1 - \mu_1)\cdots$$
  Consider for example, $\vec{X} := (X_1, X_2)$. Then
  \begin{align*}
  Var(X) \approx\;
    &\sigma^2_{X_1} \left ( \frac{\partial g(\mu)}{\partial x_1} \right )^2 + \\
    &\sigma^2_{X_2} \left ( \frac{\partial g(\mu)}{\partial x_2} \right )^2 + \\
    &2\sigma_{XY} \left ( \frac{\partial g(\mu)}{\partial x_1} \right )
    \left ( \frac{\partial g(\mu)}{\partial x_2} \right )
  \end{align*}
\end{section}

\begin{section}{Simple Random Sampling}
  Simple random sampling \textit{without replacement} means that each sample is
  not independent of another. While the mean of the simple random sample is
  still unbiased, that is $\expectation{\bar{X}} = \mu$,
  $$Cov(X_i, X_j) = -\sigma^2 / (N-1)$$
  for two different simple random samples, i.e. $i \neq j$. The variance of the
  sample mean then becomes
  $$Var(\bar{X}) = \frac{\sigma^2}{n} \left ( \frac{N - n}{N - 1} \right )$$
  The variance of the sample total is
  $$Var(T) = N^2 \left ( \frac{\sigma^2}{n} \right ) \frac{N- n}{N-1}$$
  $\sigma$ is unknown and must be estimated. 
  \begin{align*}
    s^2_{\bar{X}} &= \frac{s^2}{n} \left ( 1 - \frac{n}{N} \right ) \\
    s^2_T &= N^2 s^2_{\bar{X}}
  \end{align*}
  where $s^2 = \frac{1}{n-1}\sum^n_{i=1}(X_i - \bar{X})^2$ is the unbiased
  sample variance.
\end{section}

\begin{section}{Consistency}
  Let $\hat{\theta}_n$ be an estimate of a parameter $\theta_0$ based on a
  sample of size $n$. $\hat{\theta}_n$ is said to be consistent in probability if
  $\hat{\theta}_n$ converges in probability to $\theta_0$ as $n$ approaches
  infinity. That is, for $\epsilon > 0$,
  $$P(|\hat{\theta}_n - \theta_0| > \epsilon) \rightarrow 0 \text{ as } n
  \rightarrow \infty$$
\end{section}

\begin{section}{Fisher Information}
  \begin{align*}
    I(\theta) &= \operatorname{E} \left [
      \frac{\partial}{\partial\theta} \log f(X|\theta)
    \right ]^2 \\
      &= - \operatorname{E} \left [
        \frac{\partial^2}{\partial\theta^2} \log f(X|\theta)
    \right ]
  \end{align*}
\end{section}

\begin{section}{Large Sample Theory for MLE}
  Let $\hat{\theta}$ denote the MLE of $\theta_0$. The probability distribution
  of
  $$\sqrt{nI(\theta_0)}(\hat{\theta} - \theta_0)$$
  tends to a standard normal distribution. Therefore, the asymptotic variance of
  the MLE is
  $$\frac{1}{nI(\theta)} = - \frac{1}{\expectation{l''(\theta_0)}}$$
\end{section}

\begin{section}{Approximate Confidence Intervals}
  Confidence intervals can be approximated through the large sample theory for
  MLE by taking $\sqrt{nI(\theta_0)}(\hat{\theta}-\theta_0) \rightarrow N(0,
  1)$, as $n \rightarrow \infty$. \smallskip
  $$P \left (
    -z(\alpha/2) \leq
    \sqrt{nI(\hat{\theta})}(\hat{\theta} - \theta_0) \leq
    z(\alpha/2)
  \right ) \approx 1 - \alpha$$
\end{section}

\end{multicols*}
\end{document}
